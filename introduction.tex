This document describes--and in fact is--a program to play checkers
written in the Haskell\footnote{Haskell is a lazy functional language:
  \url{http://www.haskell.org/}} programming language.  This document
is a \emph{Literate Haskell} program, this means that its source code
is both a valid a \LaTeX{} document and a valid Haskell program.

gtk2hs\footnote{\url{http://www.haskell.org/gtk2hs/}} is used for the
GUI so it should have no problems running on Windows, OSX or any
flavour of Unix. However it has only been tested on Linux (Ubuntu
7.10 with GHC 6.6.1).

The modules that make up the program are:

\begin{description}
\item[Board] \myref{module:Board} Defines a data structure to represent
  a checkers board. It also defines data structures to represent
  locations on the board and ``location deltas''.
\item[Hops] \myref{module:Hops} Defines ``hops'' which are the moves
  which make up a turn.
\item[GameState] \myref{module:GameState} Defines a game state data
  structure, a successors function, and, a function to check if either
  side has won.
\item[Negamax] \myref{module:Negamax} Defines a generic version of the
  Negamax algorithm, not specific to the game of checkers.
\item[EvalFuns] \myref{module:EvalFuns} Defines static evaluation
  functions and functions to combine them. These can be used as
  heuristic functions for the Negamax algorithm.
\item[Strategies] \myref{module:Strategies} Uses the Negamax algorithm
  and the evaluation functions defined in \verb!EvalFuns! to create an
  AI strategy.
\item[ANNEvalFun] \myref{module:ANNEvalFun} An alternative method of
  creating evaluation functions using a multi-layer feed-forward
  Artificial Neural Network (ANN).
\item[EvolveANN] \myref{module:EvolveANN} A Genetic Algorithm to
  evolve a set of weights for the ANN defined in \verb!ANNEvalFun!.
\item[GUI] \myref{module:GUI} A GUI based on the GTK tool kit. This acts as the main module of the program.
\end{description}

\begin{samepage}
The program can be built using the GHC\footnote{The Glasgow Haskell
  Compiler: \url{http://www.haskell.org/ghc/}} Haskell Compiler but
may well work with Hugs\footnote{http://www.haskell.org/hugs/} as
well. The command to build the program for normal GUI operation is:

\begin{verbatim}
ghc -O2 --make GUI.lhs
\end{verbatim}
\end{samepage}

Before the PDF version of the program source code can be generated all
\emph{.lhs} files must be sym-linked so that they can also be accessed
using the \emph{.tex} extension. Once this is done the following
commands can be run.

\begin{verbatim}
pdflatex checkers.tex
makeindex checkers
pdflatex checkers.tex
\end{verbatim}

